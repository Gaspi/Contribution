\documentclass[12pt,a4paper,titlepage]{article}
\usepackage[utf8]{inputenc}
\usepackage[francais]{babel}

\usepackage[left=1.5cm,right=2cm,top=2cm,bottom=2cm]{geometry}
\usepackage{amsfonts}
\usepackage{hyperref}
\usepackage{graphicx}

\usepackage[space]{grffile} % Celui-là je sais plus à quoi il sert...


\newcommand{\Competence}{\textbf{Compétence} }
\newcommand{\Autorite}{\textbf{Autorité} }
\newcommand{\Charisme}{\textbf{Charisme} }

\newcommand{\n}[1]{\textbf{#1} }

\newcommand{\update}[3]{
\begin{center}
\begin{tabular}{|c|c|c|}
\hline #1 & #2 & #3 \\ \hline
\end{tabular}
\end{center}
}

\newcommand{\Cryptolog}{\emph{Cryptolog}}


\author{Gaspard Férey}
\title{A la Recherche \\ du \\ Temps Perdu}

\begin{document}

\maketitle

\section*{Règles}

Cette contribution adopte le format d'un \emph{Livre dont vous êtes le héros}. Il est constitué d'une multitude de chapitres très cours numérotés qu'il ne convient pas de lire les uns après les autres. Chaque chapitre est conclut par un choix laissé au lecteur. Pour signifier votre choix, rendez-vous simplement au chapitre indiqué et poursuivez
votre lecture.\\

Dans ce genre de livre, le lecteur est habituellement motivé par une quête que ses choix doivent permettre au personnage d'accomplir sans périr. Dans le cas de cette contribution, il s'agira de rentrer dans la peau du personnage que vous choisirez au début et d'adopter des choix cohérents avec sa vision de la gestion du temps au sein de l'entreprise.
En aucun cas cette contribution ne cherche à imposer une quelconque philosophie et la possibilité de perdre est uniquement présente afin d'encourager le lecteur à tenter une nouvelle approche de son aventure chez \Cryptolog. Cette contribution n'a pas pour vocation de tester le lecteur ou de juger la pertinence de ses choix.\\

Pour jouer, nul besoin de papier ni de crayon, réunissez simplement quelques objets sur votre bureau, ou quelques galets si vous avez la chance de lire cette contribution depuis quelques destination paradisiaque. Constituez trop piles qui représenteront les compteurs de vos trois caractéristiques:
\begin{itemize}
\item \Competence : Votre capacité à faire reconnaitre vos nombreux talents et à vous octroyer le  respect de vos collaborateur qui auront confiance vos décisions.
\item \Autorite : Votre capacité à affirmer votre position hiérarchique et à faire accepter vos décisions par égards aux responsabilités qui vous sont échues.
\item \Charisme : Votre capacité d'écoute et de compréhension de vos collègues. Utile entrer dans les bonnes grâces de vos collègues et pour les persuader que vos décisions sont les bonnes.
\end{itemize}
Avant de commencer votre aventure ces trois caractéristiques sont initialisées à \n{1}. Placez donc un unique objet dans chacun
de vos trois tas. En tête de chapitre vous pourrez lire un bandeau vous adjoignant de modifier vos scores.
\update{+1}{-1}{0}
Ceci signifie que votre \Competence vient d'augmenter de \n{1} point mais que votre \Autorite vient de diminuer de \n{1} point dans le même temps suite à votre dernier choix. En général, le chapitre justifiera ces modifications.
Si jamais l'une de ces valeurs venait à tomber à \n{0} ou à passer dans le négatif, finissez la lecture du chapitre mais n'effectuez pas le choix et rendez-vous immédiatement au...
\begin{itemize}
\item ... chapitre 10, si votre \Competence tombe à \n{0},
\item ... chapitre 20 si votre \Autorite tombe à \n{0},
\item ... chapitre 30 si votre \Charisme tombe à \n{0}.
\end{itemize}
Bon courage...

\newpage


Deux colonnes


\label{n1}
\subsection*{1}
Après des années à l'X... \\
\begin{itemize}
\item Vous sortez de l'X (normal), rendez-vous en 2.
\item Vous sortez des Mines (difficile), rendez-vous en 3.
\item Vous sortez d'une ENS (ultra hardcore), rendez-vous en 4.
\end{itemize}


\label{n2}
\subsection*{2}


\label{n3}
\subsection*{3}


\label{n4}
\subsection*{4}



2) Solide formation scientifique


Par bien des aspects, le temps s'apparente à une ressource.
En effet, on le mesure, on lui attribue un prix, le salaire, on le dépense, on l'échange, on le facture,
Sa bonne gestion dans le cadre du management des ressources humaines constitue un élément capital de la
performance d'une entreprise. C'est la matière première à l'aune de laquelle on évalue ...

Cependant (à l'instar ?) de toutes les autres ressources et matières premières, le temps ne saurait être stocké.
Il ne se capitalise pas.
Film Time Out / roman associé.







Brouillon contribution

Livre dont vous êtes le héros:

Par une douce chaleur d'été...
Vous êtes stagiaire chez Cryptolog, une jeune société de sécurité numérique spécialisée dans la signature électronique.



... Blabla ...
* Tiens, ... semble avoir besoin d'aide. Rendez-vous en 23
* Vous entendez du bruit à la machine à café. Vous vous levez discrètement. Rendez-vous en 18
* ... Rendez-vous en 37


Idées en vrac :

Le temps comme ressource:
Le temps: par nature ne peux être produit
gagner / perdre du temps : économiser (optimisation) / gaspiller

Le temps "perdu" peut être productif.

Jean Jaures

Votre solide formation scientifique et la renommée de votre école vous accorde une certaine légitimité. Votre engagement associatif vous a  (amicalité)






\end{document}