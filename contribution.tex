\documentclass[12pt,a4paper,titlepage]{article}
\usepackage[utf8]{inputenc}
\usepackage[francais]{babel}

\usepackage[left=1.5cm,right=2cm,top=2cm,bottom=2cm]{geometry}
\usepackage{amsfonts}
\usepackage{hyperref}
\usepackage{graphicx}

\usepackage[space]{grffile} % Celui-là je sais plus à quoi il sert...

\usepackage{multicol}
\setlength{\columnsep}{1cm}

\newcommand{\Competence}{\textbf{Compétence} }
\newcommand{\Autorite}{\textbf{Autorité} }
\newcommand{\Charisme}{\textbf{Charisme} }

\newcommand{\n}[1]{\textbf{#1} }

\newcommand{\update}[3]{
\begin{center}
\begin{tabular}{|c|c|c|}
\hline #1 & #2 & #3 \\ \hline
\end{tabular}
\end{center}
}

\newcommand{\Cryptolog}{\emph{Cryptolog}}


\author{Gaspard Férey}
\title{A la Recherche \\ du \\ Temps Perdu}

\begin{document}

\maketitle

\section*{Règles}

Cette contribution adopte le format d'un \emph{Livre dont vous êtes le héros}. Il s'agit d'un cours roman interactif constitué d'une multitude de chapitres très cours qu'il ne convient pas de lire les uns après les autres. Chaque chapitre estnuméroté et se conclut généralement par un choix laissé au lecteur. Pour signifier votre choix, rendez-vous simplement au chapitre indiqué et poursuivez votre lecture.\\

Dans ce genre de roman, le lecteur est habituellement motivé par une quête que ses choix doivent permettre au héros d'accomplir. Dans le cas de cette contribution, il s'agira de rentrer dans la peau du personnage que vous choisirez au début et d'adopter des choix cohérents avec sa vision de la gestion du temps au sein de l'entreprise.
En aucun cas cette contribution ne cherche à imposer une quelconque philosophie et la possibilité de perdre est uniquement présente afin d'encourager le lecteur à tenter une nouvelle approche de son aventure chez \Cryptolog. Cette contribution n'a pas pour vocation de tester le lecteur ou de juger la pertinence de ses choix.\\

Pour jouer, nul besoin de papier ou de crayon, réunissez simplement quelques objets devant vous et constituez trop piles qui représenteront les compteurs de vos trois caractéristiques:
\begin{itemize}
\item \Competence : Votre capacité à faire reconnaitre vos nombreux talents et à vous octroyer le  respect de vos collaborateur qui auront confiance vos décisions.
\item \Autorite : Votre capacité à affirmer votre position hiérarchique et à faire accepter vos décisions par égards aux responsabilités qui vous sont échues.
\item \Charisme : Votre capacité d'écoute et de compréhension de vos collègues. Utile entrer dans les bonnes grâces de vos collègues et pour les persuader que vos décisions sont les bonnes.
\end{itemize}
Avant de commencer votre aventure ces trois caractéristiques sont initialisées à \n{1}. Placez donc un unique objet dans chacun
de vos trois tas. En tête de chapitre vous pourrez lire un bandeau vous adjoignant de modifier vos scores.
\update{+1}{-1}{0}
Ceci signifie que votre \Competence vient d'augmenter de \n{1} point mais que votre \Autorite vient de diminuer de \n{1} point dans le même temps suite à votre dernier choix. En général, le chapitre justifiera ces modifications.
Si jamais l'une de ces valeurs venait à tomber à \n{0} ou à passer dans le négatif, finissez la lecture du chapitre mais n'effectuez pas le choix et rendez-vous immédiatement au...
\begin{itemize}
\item ... chapitre 10, si votre \Competence tombe à \n{0},
\item ... chapitre 20 si votre \Autorite tombe à \n{0},
\item ... chapitre 30 si votre \Charisme tombe à \n{0}.
\end{itemize}
Bon courage...

\newpage

\begin{multicols}{2}
[
\section{Premiers pas chez Cryptolog...}
Biip, biip... Biip biip... \\
Il n'est que 8h30 mais c'est plein d'entrain que vous sautez hors du lit. C'est aujourd'hui votre premier jour de stage chez Cryptolog, une jeune société de sécurité numérique spécialisée dans la signature électronique.

Vous essayez tant bien que mal de vous remémorer votre dernier entretien avec votre tuteur. Il vous avait alors décrit votre mission.
\begin{itemize}
\item "Votre mission principale sera de reprendre un projet de recherche et développement, ICARe. Il s'agit d'un outil de reconnaissance automatique de document d'identité développé intégralement au sein de Cryptolog. Vous aurez l'occasion de travailler avec des collaborateurs Tunisiens." Rendez-vous en \n{1}.
\item "Je souhaite mettre en place le systeme SCRUMS dans mon entreprise, vous serez en charge de développer un outil adapté à notre activité ainsi qu'à former le personnel à son utilisation. Vous prendrez également part au processus de recrutement." Rendez-vous en \n{3}.
\item "L'activité commerciale de Cryptolog est en train d'exploser avec la multiplication de nos clients. Vous serez le correspondant de plusieurs de ces clients chez Cryptolog." Rendez-vous en \n{2}.
\end{itemize}
]


\label{n1}
\subsection*{1}
\update{+1}{0}{0}
Votre cursus à dominante très théorique vous permet d'avoir une certain légitimité reconnue par vos collègues à votre arrivée.

\label{n2}
\subsection*{2}
\update{0}{+1}{0}
Votre stage de formation humain et militaire vous a permis de développer une confiance en vous qui vous sera indispensable à la gestion du projet qui vous est assigné.

\label{n3}
\subsection*{3}
\update{0}{0}{+1}
Votre dévouement dans la vie associative de votre école vous a rendu très chaleureux et communicatif. Le courant passe très bien lors de votre première journée. Vos collègues vous indiquent votre bureau, échangent quelques anecdotes et plaisante gaiement.

\begin{itemize}
\item Le temps passe vite et...
\end{itemize}




2) Solide formation scientifique


Par bien des aspects, le temps s'apparente à une ressource.
En effet, on le mesure, on lui attribue un prix, le salaire, on le dépense, on l'échange, on le facture,
Sa bonne gestion dans le cadre du management des ressources humaines constitue un élément capital du bon fonctionnement d'une entreprise. C'est la matière première à l'aune de laquelle on évalue sa performance.

Cependant (à l'instar ?) de toutes les autres ressources et matières premières, le temps ne saurait être stocké.
Il ne se capitalise pas.
Film Time Out / roman associé.










... Blabla ...
* Tiens, ... semble avoir besoin d'aide. Rendez-vous en 23
* Vous entendez du bruit à la machine à café. Vous vous levez discrètement. Rendez-vous en 18
* ... Rendez-vous en 37


Idées en vrac :

Le temps comme ressource:
Le temps: par nature ne peux être produit
gagner / perdre du temps : économiser (optimisation) / gaspiller

Le temps "perdu" peut être productif.

Jean Jaures

Votre solide formation scientifique et la renommée de votre école vous accorde une certaine légitimité. Votre engagement associatif vous a  (amicalité)




\label{n1}
\subsection*{1}
Après des années à l'X... \\
\begin{itemize}
\item Vous sortez de l'X (normal), rendez-vous en 1.
\item Vous sortez des Mines (difficile), rendez-vous en 2.
\item Vous sortez d'une ENS (ultra hardcore), rendez-vous en 3.
\end{itemize}



\end{multicols}

\end{document}
